\documentclass{article} % For LaTeX2e
\usepackage{nips15submit_e,times}
\usepackage{hyperref}
\usepackage{url}
%\documentstyle[nips14submit_09,times,art10]{article} % For LaTeX 2.09


\title{Higgs Machine Learning Challenge \\with Neural Network Adversarial Training}


\author{
Oren Shklarsky\\
Department of Physics\\
Simon Fraser University\\
Burnaby, BC V5A 1S6\\
\texttt{oshklarsky@sfu.ca}\\
\And
Darren Temple\\
Department of Physics\\
Simon Fraser University\\
Burnaby, BC V5A 1S6\\
\texttt{dtemple@sfu.ca}\\
\AND
Morgan Jenkins\\
Department of Computer Science\\
Simon Fraser University\\
Burnaby, BC V5A 1S6\\
\texttt{mjenkins@sfu.ca}\\
}

% The \author macro works with any number of authors. There are two commands
% used to separate the names and addresses of multiple authors: \And and \AND.
%
% Using \And between authors leaves it to \LaTeX{} to determine where to break
% the lines. Using \AND forces a linebreak at that point. So, if \LaTeX{}
% puts 3 of 4 authors names on the first line, and the last on the second
% line, try using \AND instead of \And before the third author name.

\newcommand{\fix}{\marginpar{FIX}}
\newcommand{\new}{\marginpar{NEW}}

%\nipsfinalcopy % Uncomment for camera-ready version

\begin{document}


\maketitle

\begin{abstract}
The Higgs particle was theorized in 1964 and finally discovered in 2012 data from the ATLAS and CMS experiments at CERN. This particle is the keystone of the Standard Model of particle physics, and provides the mechanism by which other particles acquire mass. Such discoveries are statistical in nature, and require vast amounts of information from the debris of particle collisions. Sifting and categorizing this data is a continual challenge, and in 2014 CERN and Kaggle invited the public to try their hand at such work, via the Higgs Machine Learning Challenge. A number of top entries, including the winner, used neural networks. An extended approach using the so-called "adversarial training" technique is presented herein.
\end{abstract}

\section{Introduction}

Particle physics arguably began with the discovery of the electron by J.J. Thomson in 18??. This began a century of subsequent rapid discoveries of new particles, next being the identification of the proton in 19?? and then the neutron in 19??, both by Ernest Rutherford. These three particles - electron, proton and neutron - are all that's needed to build the everyday chemical elements that comprise the apparent material universe. However, it was with early studies of cosmic rays that our understanding of fundamental particles began to change. One after the other more and more particles were discovered, with a broad spectrum of characteristics. Patterns slowly emerged, leading to the so-called Standard Model of particle physics, which encompasses not only particles of a material nature but also particles that transmit the very forces between the former category, thereby enabling their interactions.

As part of this framework, an altogether different particle was theorized to exist in order to explain why certain particles have mass. This came to be known as the Higgs particle, named after the author of the groundbreaking 1964 paper by one of the pioneers of this field. So began a search that spanned nearly four decades, culminating with its discovery in 2012 data from the Large Hadron Collider at the CERN lab in Geneva, as measured by the ATLAS and CMS experiments. The tell-tale signals were given by the physics properties of the Higgs particle's decay products, as the particle itself cannot be directly observed due to decaying so rapidly that it cannot travel out of its creation region and into the detector.

There are two main abilities of the experimental apparatus that allowed for the Higgs particle to be discovered, that were not had with earlier experiments. First and foremost is the particle collision energy, and secondly is the ability of the detector equipment to record and analyze data. Every particle has a characteristic mass energy, and this energy must be met by a device such as the LHC in order for the particle to be created. Energy is the raw ingredient of all particles, in other words, and every particle requires a certain amount of this ingredient. The Higgs particle happens to be a relatively massive particle compared to a number of others, and therefore requires a relatively large amount of energy. A technological evolution over many years and many generations of device was necessary to develop the LHC, which is powerful enough to provide the Higgs creation energy.

Once created, a high mass particle such as the Higgs rapidly decays into other, lighter particles. It is these decay products that are detected by the machine surrounding the creation point, which are evaluated for such properties as momentum and charge, based on their trajectories through the machine, leading to their classification. Originally, detectors were comprised of some continuous medium that would reveal a particle track that could be photographed and measured. Things have since come a long way, with current detectors being formed of multiple layers of various electronics, each sampling the outgoing decay products in different ways. Despite the LHC providing an appropriate energy range for the creation of Higgs particles, their production is still exceedingly rare compared with the production of a number of other particles. An exceedingly large number of creation events is therefore required to compensate for the rarity, and the particle tracks simply cannot be assessed individually as used to be possible in photographic detectors from previous generations. The vast stream of electronic readout must be dealt with computationally and statistically, and a great deal of time and effort is spent by scientists determining how best to seek the signal of interest.

Machine learning has inevitably become an invaluable tool in this task, which, following the reconstruction of particle properties from the raw signal, is simply one of classification. Namely, it is a question of signal versus background. In the case of the Higgs particle, its various decay channels are not unique, and are in fact shared by other other particles such as the Z particle, which is the carrier of the neutral weak force. Seeing a certain set of decay products does not then give indisputable proof of one particular parent particle or the other. But there are nonetheless subtle signs within the data when viewed \textit{en masse}, which machine learning algorithms can be trained to evaluate.

In 2014, CERN scientists collaborated with the Kaggle machine learning organization, developing an open competition for the public to try and develop a machine learning classifier to distinguish the Higgs particle from the Z particle in the tau-tau decay channel. The tau is a heavier version of the electron, and these tau particles also decay into a range of products that enter the detector. It is therefore these decay products that are used to infer the taus, and the taus are then used to infer either the Higgs or Z particles. This Kaggle challenge was the most popular to date, with more than ??? entries from numerous participants from around the world, some working within the physics community, others simply interested in computing. The winning entry came from a sole programmer with no advanced physics training, who developed a neural network model that was clearly superior to anything else.

Many technical fields are developing incredibly fast at present, and machine learning is no exception. Even just more than one year on from the Higgs machine learning challenge, new ideas and techniques have become available. This project seeks to explore one such technique that has started to generate a lot of interest of late, namely that of adversarial training, and to apply it to the Higgs machine learning challenge.

\subsection{Style}

Papers to be submitted to NIPS 2015 must be prepared according to the
instructions presented here. Papers may be only up to eight pages long,
including figures. Since 2009 an additional ninth page \textit{containing only
cited references} is allowed. Papers that exceed nine pages will not be
reviewed, or in any other way considered for presentation at the conference.
%This is a strict upper bound. 

Please note that this year we have introduced automatic line number generation
into the style file (for \LaTeXe and Word versions). This is to help reviewers
refer to specific lines of the paper when they make their comments. Please do
NOT refer to these line numbers in your paper as they will be removed from the
style file for the final version of accepted papers.

The margins in 2015 are the same as since 2007, which allow for $\approx 15\%$
more words in the paper compared to earlier years. We are also again using 
double-blind reviewing. Both of these require the use of new style files.

Authors are required to use the NIPS \LaTeX{} style files obtainable at the
NIPS website as indicated below. Please make sure you use the current files and
not previous versions. Tweaking the style files may be grounds for rejection.

%% \subsection{Double-blind reviewing}

%% This year we are doing double-blind reviewing: the reviewers will not know 
%% who the authors of the paper are. For submission, the NIPS style file will 
%% automatically anonymize the author list at the beginning of the paper.

%% Please write your paper in such a way to preserve anonymity. Refer to
%% previous work by the author(s) in the third person, rather than first
%% person. Do not provide Web links to supporting material at an identifiable
%% web site.

%%\subsection{Electronic submission}
%%
%% \textbf{THE SUBMISSION DEADLINE IS June 5, 2015. SUBMISSIONS MUST BE LOGGED BY
%% 23:00, June 5, 2015, UNIVERSAL TIME}

%% You must enter your submission in the electronic submission form available at
%% the NIPS website listed above. You will be asked to enter paper title, name of
%% all authors, keyword(s), and data about the contact
%% author (name, full address, telephone, fax, and email). You will need to
%% upload an electronic (postscript or pdf) version of your paper.

%% You can upload more than one version of your paper, until the
%% submission deadline. We strongly recommended uploading your paper in
%% advance of the deadline, so you can avoid last-minute server congestion.
%%
%% Note that your submission is only valid if you get an e-mail
%% confirmation from the server. If you do not get such an e-mail, please
%% try uploading again. 


\subsection{Retrieval of style files}

The style files for NIPS and other conference information are available on the World Wide Web at
\begin{center}
   \url{http://www.nips.cc/}
\end{center}
The file \verb+nips2015.pdf+ contains these 
instructions and illustrates the
various formatting requirements your NIPS paper must satisfy. \LaTeX{}
users can choose between two style files:
\verb+nips15submit_09.sty+ (to be used with \LaTeX{} version 2.09) and
\verb+nips15submit_e.sty+ (to be used with \LaTeX{}2e). The file
\verb+nips2015.tex+ may be used as a ``shell'' for writing your paper. All you
have to do is replace the author, title, abstract, and text of the paper with
your own. The file
\verb+nips2015.rtf+ is provided as a shell for MS Word users.

The formatting instructions contained in these style files are summarized in
sections \ref{gen_inst}, \ref{headings}, and \ref{others} below.

%% \subsection{Keywords for paper submission}
%% Your NIPS paper can be submitted with any of the following keywords (more than one keyword is possible for each paper):

%% \begin{verbatim}
%% Bioinformatics
%% Biological Vision
%% Brain Imaging and Brain Computer Interfacing
%% Clustering
%% Cognitive Science
%% Control and Reinforcement Learning
%% Dimensionality Reduction and Manifolds
%% Feature Selection
%% Gaussian Processes
%% Graphical Models
%% Hardware Technologies
%% Kernels
%% Learning Theory
%% Machine Vision
%% Margins and Boosting
%% Neural Networks
%% Neuroscience
%% Other Algorithms and Architectures
%% Other Applications
%% Semi-supervised Learning
%% Speech and Signal Processing
%% Text and Language Applications

%% \end{verbatim}

\section{General formatting instructions}
\label{gen_inst}

The text must be confined within a rectangle 5.5~inches (33~picas) wide and
9~inches (54~picas) long. The left margin is 1.5~inch (9~picas).
Use 10~point type with a vertical spacing of 11~points. Times New Roman is the
preferred typeface throughout. Paragraphs are separated by 1/2~line space,
with no indentation.

Paper title is 17~point, initial caps/lower case, bold, centered between
2~horizontal rules. Top rule is 4~points thick and bottom rule is 1~point
thick. Allow 1/4~inch space above and below title to rules. All pages should
start at 1~inch (6~picas) from the top of the page.

%The version of the paper submitted for review should have ``Anonymous Author(s)'' as the author of the paper.

For the final version, authors' names are
set in boldface, and each name is centered above the corresponding
address. The lead author's name is to be listed first (left-most), and
the co-authors' names (if different address) are set to follow. If
there is only one co-author, list both author and co-author side by side.

Please pay special attention to the instructions in section \ref{others}
regarding figures, tables, acknowledgments, and references.

\section{Headings: first level}
\label{headings}

First level headings are lower case (except for first word and proper nouns),
flush left, bold and in point size 12. One line space before the first level
heading and 1/2~line space after the first level heading.

\subsection{Headings: second level}

Second level headings are lower case (except for first word and proper nouns),
flush left, bold and in point size 10. One line space before the second level
heading and 1/2~line space after the second level heading.

\subsubsection{Headings: third level}

Third level headings are lower case (except for first word and proper nouns),
flush left, bold and in point size 10. One line space before the third level
heading and 1/2~line space after the third level heading.

\section{Citations, figures, tables, references}
\label{others}

These instructions apply to everyone, regardless of the formatter being used.

\subsection{Citations within the text}

Citations within the text should be numbered consecutively. The corresponding
number is to appear enclosed in square brackets, such as [1] or [2]-[5]. The
corresponding references are to be listed in the same order at the end of the
paper, in the \textbf{References} section. (Note: the standard
\textsc{Bib\TeX} style \texttt{unsrt} produces this.) As to the format of the
references themselves, any style is acceptable as long as it is used
consistently.

As submission is double blind, refer to your own published work in the 
third person. That is, use ``In the previous work of Jones et al.\ [4]'',
not ``In our previous work [4]''. If you cite your other papers that
are not widely available (e.g.\ a journal paper under review), use
anonymous author names in the citation, e.g.\ an author of the
form ``A.\ Anonymous''. 


\subsection{Footnotes}

Indicate footnotes with a number\footnote{Sample of the first footnote} in the
text. Place the footnotes at the bottom of the page on which they appear.
Precede the footnote with a horizontal rule of 2~inches
(12~picas).\footnote{Sample of the second footnote}

\subsection{Figures}

All artwork must be neat, clean, and legible. Lines should be dark
enough for purposes of reproduction; art work should not be
hand-drawn. The figure number and caption always appear after the
figure. Place one line space before the figure caption, and one line
space after the figure. The figure caption is lower case (except for
first word and proper nouns); figures are numbered consecutively.

Make sure the figure caption does not get separated from the figure.
Leave sufficient space to avoid splitting the figure and figure caption.

You may use color figures. 
However, it is best for the
figure captions and the paper body to make sense if the paper is printed
either in black/white or in color.
\begin{figure}[h]
\begin{center}
%\framebox[4.0in]{$\;$}
\fbox{\rule[-.5cm]{0cm}{4cm} \rule[-.5cm]{4cm}{0cm}}
\end{center}
\caption{Sample figure caption.}
\end{figure}

\subsection{Tables}

All tables must be centered, neat, clean and legible. Do not use hand-drawn
tables. The table number and title always appear before the table. See
Table~\ref{sample-table}.

Place one line space before the table title, one line space after the table
title, and one line space after the table. The table title must be lower case
(except for first word and proper nouns); tables are numbered consecutively.

\begin{table}[t]
\caption{Sample table title}
\label{sample-table}
\begin{center}
\begin{tabular}{ll}
\multicolumn{1}{c}{\bf PART}  &\multicolumn{1}{c}{\bf DESCRIPTION}
\\ \hline \\
Dendrite         &Input terminal \\
Axon             &Output terminal \\
Soma             &Cell body (contains cell nucleus) \\
\end{tabular}
\end{center}
\end{table}

\section{Final instructions}
Do not change any aspects of the formatting parameters in the style files.
In particular, do not modify the width or length of the rectangle the text
should fit into, and do not change font sizes (except perhaps in the
\textbf{References} section; see below). Please note that pages should be
numbered.

\section{Preparing PostScript or PDF files}

Please prepare PostScript or PDF files with paper size ``US Letter'', and
not, for example, ``A4''. The -t
letter option on dvips will produce US Letter files.

Fonts were the main cause of problems in the past years. Your PDF file must
only contain Type 1 or Embedded TrueType fonts. Here are a few instructions
to achieve this.

\begin{itemize}

\item You can check which fonts a PDF files uses.  In Acrobat Reader,
select the menu Files$>$Document Properties$>$Fonts and select Show All Fonts. You can
also use the program \verb+pdffonts+ which comes with \verb+xpdf+ and is
available out-of-the-box on most Linux machines.

\item The IEEE has recommendations for generating PDF files whose fonts
are also acceptable for NIPS. Please see
\url{http://www.emfield.org/icuwb2010/downloads/IEEE-PDF-SpecV32.pdf}

\item LaTeX users:

\begin{itemize}

\item Consider directly generating PDF files using \verb+pdflatex+
(especially if you are a MiKTeX user). 
PDF figures must be substituted for EPS figures, however.

\item Otherwise, please generate your PostScript and PDF files with the following commands:
\begin{verbatim} 
dvips mypaper.dvi -t letter -Ppdf -G0 -o mypaper.ps
ps2pdf mypaper.ps mypaper.pdf
\end{verbatim}

Check that the PDF files only contains Type 1 fonts. 
%For the final version, please send us both the Postscript file and
%the PDF file. 

\item xfig "patterned" shapes are implemented with 
bitmap fonts.  Use "solid" shapes instead. 
\item The \verb+\bbold+ package almost always uses bitmap
fonts.  You can try the equivalent AMS Fonts with command
\begin{verbatim}
\usepackage[psamsfonts]{amssymb}
\end{verbatim}
 or use the following workaround for reals, natural and complex: 
\begin{verbatim}
\newcommand{\RR}{I\!\!R} %real numbers
\newcommand{\Nat}{I\!\!N} %natural numbers 
\newcommand{\CC}{I\!\!\!\!C} %complex numbers
\end{verbatim}

\item Sometimes the problematic fonts are used in figures
included in LaTeX files. The ghostscript program \verb+eps2eps+ is the simplest
way to clean such figures. For black and white figures, slightly better
results can be achieved with program \verb+potrace+.
\end{itemize}
\item MSWord and Windows users (via PDF file):
\begin{itemize}
\item Install the Microsoft Save as PDF Office 2007 Add-in from
\url{http://www.microsoft.com/downloads/details.aspx?displaylang=en\&familyid=4d951911-3e7e-4ae6-b059-a2e79ed87041}
\item Select ``Save or Publish to PDF'' from the Office or File menu
\end{itemize}
\item MSWord and Mac OS X users (via PDF file):
\begin{itemize}
\item From the print menu, click the PDF drop-down box, and select ``Save
as PDF...''
\end{itemize}
\item MSWord and Windows users (via PS file):
\begin{itemize}
\item To create a new printer
on your computer, install the AdobePS printer driver and the Adobe Distiller PPD file from
\url{http://www.adobe.com/support/downloads/detail.jsp?ftpID=204} {\it Note:} You must reboot your PC after installing the
AdobePS driver for it to take effect.
\item To produce the ps file, select ``Print'' from the MS app, choose
the installed AdobePS printer, click on ``Properties'', click on ``Advanced.''
\item Set ``TrueType Font'' to be ``Download as Softfont''
\item Open the ``PostScript Options'' folder
\item Select ``PostScript Output Option'' to be ``Optimize for Portability''
\item Select ``TrueType Font Download Option'' to be ``Outline''
\item Select ``Send PostScript Error Handler'' to be ``No''
\item Click ``OK'' three times, print your file.
\item Now, use Adobe Acrobat Distiller or ps2pdf to create a PDF file from
the PS file. In Acrobat, check the option ``Embed all fonts'' if
applicable.
\end{itemize}

\end{itemize}
If your file contains Type 3 fonts or non embedded TrueType fonts, we will
ask you to fix it. 

\subsection{Margins in LaTeX}
 
Most of the margin problems come from figures positioned by hand using
\verb+\special+ or other commands. We suggest using the command
\verb+\includegraphics+
from the graphicx package. Always specify the figure width as a multiple of
the line width as in the example below using .eps graphics
\begin{verbatim}
   \usepackage[dvips]{graphicx} ... 
   \includegraphics[width=0.8\linewidth]{myfile.eps} 
\end{verbatim}
or % Apr 2009 addition
\begin{verbatim}
   \usepackage[pdftex]{graphicx} ... 
   \includegraphics[width=0.8\linewidth]{myfile.pdf} 
\end{verbatim}
for .pdf graphics. 
See section 4.4 in the graphics bundle documentation (\url{http://www.ctan.org/tex-archive/macros/latex/required/graphics/grfguide.ps}) 
 
A number of width problems arise when LaTeX cannot properly hyphenate a
line. Please give LaTeX hyphenation hints using the \verb+\-+ command.


\subsubsection*{Acknowledgments}

Use unnumbered third level headings for the acknowledgments. All
acknowledgments go at the end of the paper. Do not include 
acknowledgments in the anonymized submission, only in the 
final paper. 

\subsubsection*{References}

References follow the acknowledgments. Use unnumbered third level heading for
the references. Any choice of citation style is acceptable as long as you are
consistent. It is permissible to reduce the font size to `small' (9-point) 
when listing the references. {\bf Remember that this year you can use
a ninth page as long as it contains \emph{only} cited references.}

\small{
[1] Alexander, J.A. \& Mozer, M.C. (1995) Template-based algorithms
for connectionist rule extraction. In G. Tesauro, D. S. Touretzky
and T.K. Leen (eds.), {\it Advances in Neural Information Processing
Systems 7}, pp. 609-616. Cambridge, MA: MIT Press.

[2] Bower, J.M. \& Beeman, D. (1995) {\it The Book of GENESIS: Exploring
Realistic Neural Models with the GEneral NEural SImulation System.}
New York: TELOS/Springer-Verlag.

[3] Hasselmo, M.E., Schnell, E. \& Barkai, E. (1995) Dynamics of learning
and recall at excitatory recurrent synapses and cholinergic modulation
in rat hippocampal region CA3. {\it Journal of Neuroscience}
{\bf 15}(7):5249-5262.
}

\end{document}
